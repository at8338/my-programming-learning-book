\documentclass[11pt,a4j]{jarticle}
% 利用するパッケージの選択
\usepackage{amsmath,amssymb}
\usepackage{bm}
\usepackage[dvipdfmx]{graphicx}
\usepackage{ascmac}
\usepackage{color}
% 余白の設定
\usepackage[top=20truemm,bottom=20truemm,left=10truemm,right=10truemm]{geometry}

% listings.styの設定 要jlistings.sty
\usepackage{listings,jlisting}
\usepackage{courier}
\lstset{
  language={Caml},% 使用言語
  basicstyle={\ttfamily},% 書体の設定
  identifierstyle={\small},% 
  commentstyle={\small\itshape\color[cmyk]{1,0.4,1,0}},% 注釈の書体
  keywordstyle={\small\bfseries\color[cmyk]{0,1,0,0}},% キーワードの書体
  ndkeywordstyle={\small},% 
  stringstyle={\small\ttfamily\color[rgb]{0,0,1}},
  frame={single},
  breaklines=true,% 行が長い時の改行
  columns=[l]{fullflexible},%
  numbers=left,% 
  xrightmargin=0zw,%
  xleftmargin=3zw,%
  numberstyle={\scriptsize},%
  stepnumber=1,% 行番号増分
  numbersep=1zw,
  lineskip=-0.5ex,
  showstringspaces=false % 空白に文字を表示させない
}
\usepackage{lastpage}
\usepackage{fancyhdr}

\makeatletter
%%%%%%%%%%%%%%%%%%%%%%%%%%%%%%%%%%%%%%%%%%%%%%%%%%%%%%%%%%%%%%%%
%% 要編集
\title{第3回}
\author{201311367 佐藤良祐}
%%%%%%%%%%%%%%%%%%%%%%%%%%%%%%%%%%%%%%%%%%%%%%%%%%%%%%%%%%%%%%%%
\pagestyle{fancy}

% headers & footers
\lhead{ソフトウェア技法 \@title \@author}
\chead{}
\rhead{}
\lfoot{}
\cfoot{\thepage /\pageref{LastPage}}
\rfoot{}
\renewcommand{\headrulewidth}{0pt}
\renewcommand{\footrulewidth}{0pt}
\makeatother

\begin{document}
\subsection*{課題3.1}
\lstinputlisting{./kadai31.ml}
実行結果
\begin{lstlisting}
# num_of ([4; 3; 2; 3; 4; 3], 3);;
- : int = 3
# num_of' ([4; 3; 2; 3; 4; 3], 3);;
- : int = 3
# num_of ([4; 3; 2; 3; 4; 3], 4);;
- : int = 2
# num_of' ([4; 3; 2; 3; 4; 3], 4);;
- : int = 2
\end{lstlisting}
\subsection*{課題3.2}
\lstinputlisting{./kadai32.ml}
実行結果
\begin{lstlisting}
# sum_pair' [-2; 0; 3; -1; 2; 1];;
- : int * int = (-3, 6)
# sum_pair' [1; 2; 3; -4; 5; 6];;
- : int * int = (-4, 17)
# sum_pair' [-2; 0; 3; 1; 2; 1];;
- : int * int = (-2, 7)
\end{lstlisting}
\newpage
\subsection*{課題3.3}
\lstinputlisting{./kadai33.ml}
実行結果
\begin{lstlisting}
# msort [4; 5; 2; 1];;
- : int list = [1; 2; 4; 5]
# msort [10; 9; 8; 7; 6; 5; 4; 3; 2; 1];;
- : int list = [1; 2; 3; 4; 5; 6; 7; 8; 9; 10] 
\end{lstlisting}
\newpage

\subsection*{課題3.4}
\lstinputlisting{./kadai34.ml}
実行結果
\begin{lstlisting}
# fib 10;;
- : int = 55
# fib 20;;
- : int = 6765
# fib 30;;
- : int = 832040
# fib 40;;
- : int = 102334155
# fib2 10;;
- : int * int = (55, 89)
# fib2 11;;
- : int * int = (89, 144)
# fib2 12;;
- : int * int = (144, 233)
\end{lstlisting}
\end{document}