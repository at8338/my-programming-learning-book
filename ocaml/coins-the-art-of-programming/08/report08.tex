\documentclass[11pt,a4j]{jarticle}
% 利用するパッケージの選択
\usepackage{amsmath,amssymb}
\usepackage{bm}
\usepackage[dvipdfmx]{graphicx}
\usepackage{ascmac}
\usepackage{color}
% 余白の設定
\usepackage[top=20truemm,bottom=20truemm,left=10truemm,right=10truemm]{geometry}

% listings.styの設定 要jlistings.sty
\usepackage{listings,jlisting}
\usepackage{courier}
\lstset{
  language={Caml},% 使用言語
  basicstyle={\ttfamily},% 書体の設定
  identifierstyle={\small},% 
  commentstyle={\small\itshape\color[cmyk]{1,0.4,1,0}},% 注釈の書体
  keywordstyle={\small\bfseries\color[cmyk]{0,1,0,0}},% キーワードの書体
  ndkeywordstyle={\small},% 
  stringstyle={\small\ttfamily\color[rgb]{0,0,1}},
  frame={single},
  breaklines=true,% 行が長い時の改行
  columns=[l]{fullflexible},%
  numbers=left,% 
  xrightmargin=0zw,%
  xleftmargin=3zw,%
  numberstyle={\scriptsize},%
  stepnumber=1,% 行番号増分
  numbersep=1zw,
  lineskip=-0.5ex,
  showstringspaces=false % 空白に文字を表示させない
}
\usepackage{lastpage}
\usepackage{fancyhdr}

\makeatletter
%%%%%%%%%%%%%%%%%%%%%%%%%%%%%%%%%%%%%%%%%%%%%%%%%%%%%%%%%%%%%%%%
%% 要編集
\title{第8回}
\author{201311367 佐藤良祐}
\西暦
\date{\today}
%%%%%%%%%%%%%%%%%%%%%%%%%%%%%%%%%%%%%%%%%%%%%%%%%%%%%%%%%%%%%%%%
\pagestyle{fancy}

% headers & footers
\lhead{ソフトウェア技法 \@title \@author \@date}
\chead{}
\rhead{}
\lfoot{}
\cfoot{\thepage /\pageref{LastPage}}
\rfoot{}
\renewcommand{\headrulewidth}{0pt}
\renewcommand{\footrulewidth}{0pt}
\makeatother

\begin{document}
\section*{課題8.1}
(1), (2)を両方まとめてのせる.

\lstinputlisting[]{./kadai81.ml}

\subsection*{(1)}
\begin{lstlisting}
# exp2string (PRIMexp (PLUSop, INTexp 1, FLOATexp 2.0));;
- : string = "(1+2.)"
# exp2string (LETexp ("x", INTexp 1, PRIMexp (MULop, VARexp "x", VARexp "x")));;
- : string = "let x=1 in (x*x)"
\end{lstlisting}
\subsection*{(2)}
\begin{lstlisting}
# eval [] (PRIMexp (PLUSop, INTexp 1, FLOATexp 1.5));;
- : value = FLOATval 2.5
# eval [] (LETexp ("x", INTexp 2, PRIMexp (MULop, VARexp "x", VARexp "x")));;
- : value = INTval 4
\end{lstlisting}
\newpage
\section*{課題8.2}
実行サンプルもソースコードに埋め込んであるので以下のようにすぐ実行できる.
\lstinputlisting[]{./kadai82.ml}
\begin{lstlisting}
# eval [] exp1;;
- : value = INTval 3
# eval [] (sum 4);;
- : value = INTval 10
\end{lstlisting}
\end{document}